\documentclass[12pt]{article}
\usepackage[utf8]{inputenc}
\usepackage{lscape}
\usepackage{graphicx}
\usepackage{stfloats}
\usepackage{float}
\usepackage{import}
\usepackage{adjustbox}
\usepackage{hyperref}
\usepackage{apacite}
\usepackage{fancyhdr}
\usepackage{amsmath}


\pagestyle{fancy}
\lhead{Niklas Lundberg}
\rhead{inaule-6@student.ltu.se}

\setlength{\parindent}{0em}
\setlength{\parskip}{1em}

\title{Home Exam D7050E}
\author{Niklas Lundberg \\ inaule-6@student.ltu.se}
\date{\today}

\begin{document}

    \maketitle
    \newpage
	
    \section{Rubigo-lang Syntax}
    \subsection{EBNF}
    	\begin{verbatim}
(* Definition of Program *)
Program = Module ;


(* Definition of Module *)
Module = { Statement } ;


(* Definition of Statement *)
    Function = "fn", Identifier, "(", [ Identifier, ":",
        Type_Declaration, { ",", Identifier, ":", 
        Type_Declaration } ], ")", "->", Type_Declaration,
        "{", { Statement }, "}" ;
    While = "while", Expression, "{", { Statement }, "}" ;
    If = "if", Expression, "{", { Statement }, "}", [ "else",
        "{", { Statement }, "}" ] ;
    Let = "let", [ Mutable ], Identifier, ":", Type_Declaration,
        "=", Expression, ";" ;
    Assignment = [ Dereference ], Identifier, "=",
        Expression, ";" ;
    Return = "return", Expression, ";" ;
    Function_Call = E_Function_Call,  ";" ;
    Statement = Function | While | If | Let | Assignment |
        Return | Function_Call ;


(* Definition of Expression (E stands for expression) *)
    E_Binary_Operation = Expression, Binary_Operator,
        Expression ;
    E_Unary_Operation = Unary_Operator, Expression ;
    E_Function_Call = Identifier, "(", [ Expression, { ",",
        Expression } ], ")" ;
    E_Variable = Identifier ;
    E_Borrowed = "&", Expression ;
    E_Dereferenced = "*", Expression ;
    E_Mutable = Mutable, Expression ;
    Expression = E_Binary_Operation | E_Unary_Operation |
        E_Function_Call | E_Variable | Literal | E_Borrowed |
        E_Dereferenced | E_Mutable ; 


(* Definition of Type_Declaration *)
    Mutable = "mut" ;
    Borrow = "&" ;
    Dereference = "*" ;
    Type_Declaration = [ Borrow ], [ Mutable ], Type ;


(* Definition of Binary_Operator *)
    Add = "+" ;
    Sub = "-" ;
    Div = "/" ;
    Multi = "*" ;
    Mod = "%" ;
    And = "&&" ;
    Or = "||" ;
    Equal = "==" ;
    Not_Equal = "!=" ;
    Less_Then = "<" ;
    Larger_Then = ">" ;
    Less_Equal_Then = "<=" ;
    Larger_Equal_Then = ">=" ;
    Binary_Operator = Add | Sub ¦ Div ¦ Multi | Mod | And |
        Or | Not | Equal | Not_Equal | Less_Then |
        Larger_Then | Less_Equal_Then | Larger_Equal_Then ;


(* Definition of Unary Operator *)
    Sub = "+" ;
    Not = "!" ;
    Unary_Operator = Sub | Not ;


(* Definition of Literal (L stands for literal) *)
    L_I32 = Integer ;
    L_F32 = Integer, ".", Natural_Number ;
    L_Bool = True | False ;
    L_Char = "'", Character, "'" ;
    L_String = """, { Character }, """ ;  
    Literal = L_I32 | L_F32 | L_Bool | L_Char | L_String;


(* Definition of Type (T stands for type) *)
    T_Int32 = "i32" ;
    T_Float32 = "f32" ;
    T_Bool = "bool" ;
    T_Char = "Char";
    T_String = "String" ;  
    Type = T_Int32 | T_Float32 | T_Boolean | T_Char |
        T_String ;


(* General definitions *)
    Digit_Excluding_zero = r[1-9] ;
    Digit = "0" | Digit_Excluding_Zero ;
    Natural_Number = Digit_Excluding_Zero, { Digit } ;
    Integer = "0" | [ "-" ], Natural_Number ;
    Letter = r[ a-ö ] ;
    Symbol = "[" | "]" | "{" | "}" | "(" | ")" | "<" | ">"
        | "'" | '"' | "=" | "|" | "." | "," | ";" | "_" | "-" ;
    Character = Letter | Symbol | " " ;
    Identifier = ( Letter | "_" ), { Letter | "_" } ;
	\end{verbatim}

    \subsection{Example explination}
 	TODO

    \subsection{Solution compared to requirements}
 	TODO


    \section{Rubigo-lang Semantics}
    \subsection{SOS}
	General Defenitions:
    	\begin{align*}
		i &= \text{Integer} \\
		f &= \text{Float} \\
		n &\in \{i, f\} \\
		b &= \text{Boolean} \\
		v &\in \{n, b\} \\
		uop &= \text{Unary Operator} \\
		bop &= \text{Binary Operator} \\
		x &= \text{Variable} \\
		p &= \text{Pointer} \\
		e &= \text{Expression} \\
		stmt &= \text{Statement} \\
		\sigma &= \text{State/Memory} \\
		fc &= \text{Function Call} \\
	\end{align*}

	Program:
	\begin{align*}
		\frac{\langle stmt_1, \sigma \rangle\Downarrow \langle void, \sigma' \rangle}
		{\langle stmt_1;stmt_2;\cdots;stmt_n, \sigma \rangle\Downarrow \langle stmt_2;\cdots;stmt_n, \sigma' \rangle}
	\end{align*}

	Block:
	\begin{align*}
		\frac{\langle stmt_1, \sigma \rangle\Downarrow \langle void, \sigma' \rangle}
		{\langle stmt_1;stmt_2;\cdots;stmt_n, \sigma \rangle\Downarrow \langle stmt_2;\cdots;stmt_n, \sigma' \rangle}
	\end{align*}
	\begin{align*}
		\frac{\langle stmt_1, \sigma \rangle\Downarrow \langle v, \sigma' \rangle}
		{\langle stmt_1;stmt_2;\cdots;stmt_n, \sigma \rangle\Downarrow \langle v, \sigma' \rangle}
	\end{align*}

	Statement:
	\begin{align*}
		\frac{\langle stmt, \sigma \rangle\Downarrow \langle \textbf{void}, \sigma' \rangle}
		{\langle stmt, \sigma \rangle\Downarrow \sigma'}
	\end{align*}
	\begin{align*}
		\frac{\langle stmt, \sigma \rangle\Downarrow \langle v, \sigma' \rangle}
		{\langle stmt, \sigma \rangle\Downarrow \langle v, \sigma' \rangle}
	\end{align*}

	
	While:
	\begin{align*}
		\frac{\langle e, \sigma \rangle\Downarrow \langle \textbf{true}, \sigma' \rangle \ \
		\langle block, \sigma' \rangle\Downarrow \sigma''}
		{\langle \textbf{while } e \textbf{ do } block, \sigma \rangle\Downarrow \langle \textbf{while } e \textbf{ do } block, \sigma'' \rangle}
	\end{align*}
	\begin{align*}
		\frac{\langle e, \sigma \rangle\Downarrow \langle \textbf{false}, \sigma' \rangle}
		{\langle \textbf{while } e \textbf{ do } block, \sigma \rangle\Downarrow \sigma'}
	\end{align*}

	If:
	\begin{align*}
		\frac{\langle e, \sigma \rangle\Downarrow \langle \textbf{true}, \sigma' \rangle \ \
		\langle block_1, \sigma' \rangle\Downarrow \sigma''}
		{\langle \textbf{if } e \textbf{ then } block_1 \textbf{ else } block_2, \sigma \rangle\Downarrow \sigma''}
	\end{align*}
	\begin{align*}
		\frac{\langle e, \sigma \rangle\Downarrow \langle \textbf{false}, \sigma' \rangle \ \
		\langle block_2, \sigma' \rangle\Downarrow \sigma''}
		{\langle \textbf{if } e \textbf{ then } block_1 \textbf{ else } block_2, \sigma \rangle\Downarrow \sigma''}
	\end{align*}

	Return:
	\begin{align*}
		\frac{\langle e, \sigma \rangle\Downarrow \langle v, \sigma' \rangle}
		{\langle \textbf{return}\ e, \sigma \rangle\Downarrow \langle v, \sigma' \rangle}
	\end{align*}

	Let/Assignment:
	\begin{align*}
		\frac{\langle e, \sigma \rangle\Downarrow \langle v, \sigma' \rangle}
		{\langle x := e, \sigma \rangle\Downarrow \langle \sigma' [x := v] \rangle}
	\end{align*}
	\begin{align*}
		\frac{\langle e, \sigma \rangle\Downarrow \langle p, \sigma' \rangle}
		{\langle x := e, \sigma \rangle\Downarrow \langle \sigma' [x := p] \rangle}
	\end{align*}

	Function Call:
	\begin{align*}
		\frac{\langle fc, \sigma \rangle\Downarrow \langle \textbf{void}, \sigma' \rangle}
		{\langle fc, \sigma \rangle\Downarrow \sigma'}
	\end{align*}
	\begin{align*}
		\frac{\langle fc, \sigma \rangle\Downarrow \langle v, \sigma' \rangle}
		{\langle fc, \sigma \rangle\Downarrow \langle v, \sigma' \rangle}
	\end{align*}

	Expression:
	\begin{align*}
		\frac{\langle e, \sigma \rangle\Downarrow \langle v, \sigma' \rangle}
		{\langle e, \sigma \rangle\Downarrow \langle v, \sigma' \rangle}
	\end{align*}
	\begin{align*}
		\frac{\langle e, \sigma \rangle\Downarrow \langle p, \sigma' \rangle}
		{\langle e, \sigma \rangle\Downarrow \langle p, \sigma' \rangle}
	\end{align*}

	Binary Operations:
	\begin{align*}
		\frac{\langle e_1, \sigma \rangle\Downarrow \langle v_1, \sigma' \rangle \ \
		\langle e_2, \sigma' \rangle\Downarrow \langle v_2, \sigma'' \rangle \ \
		\langle v_1\ \textbf{bop}\ v_2, \sigma'' \rangle\Downarrow \langle v_3, \sigma'' \rangle}
		{\langle e_1\ \textbf{bop}\ e_2, \sigma \rangle\Downarrow \langle v_3, \sigma'' \rangle}
	\end{align*}
	
	Unary Operations:
	\begin{align*}
		\frac{\langle e, \sigma \rangle\Downarrow \langle v, \sigma' \rangle \ \
		\langle uop\ v, \sigma' \rangle\Downarrow \langle v', \sigma' \rangle}
		{\langle uop\ e, \sigma \rangle\Downarrow \langle v', \sigma' \rangle}
	\end{align*}

	Borrow Variable:
	\begin{align*}
		\frac{}
		{\langle \&x, \sigma \rangle\Downarrow \langle p, \sigma \rangle}
	\end{align*}

	Dereference Pointer:
	\begin{align*}
		\frac{\langle \*p, \sigma \rangle\Downarrow \langle p', \sigma \rangle}
		{\langle \*p, \sigma \rangle\Downarrow \langle p', \sigma \rangle}
	\end{align*}

	\begin{align*}
		\frac{\langle \*p, \sigma \rangle\Downarrow \langle v, \sigma \rangle}
		{\langle \*p, \sigma \rangle\Downarrow \langle v, \sigma \rangle}
	\end{align*}	

	Variable:
	\begin{align*}
		\frac{}
		{\langle x, \sigma \rangle\Downarrow \langle v, \sigma \rangle}
	\end{align*}

	Value:
	\begin{align*}
		\frac{}
		{\langle v, \sigma \rangle\Downarrow \langle v, \sigma \rangle}
	\end{align*}

    \subsection{Example explination}
    	TODO
    
    \subsection{Solution compared to requirements}
 	TODO





    \section{Rubigo-lang Type Checker}	
    	\subsection{Type Checking Rules}
 		TODO

	\subsection{Example explination}
 		TODO
    	
	\subsection{Solution compared to requirements}
 		TODO
    




    \section{Rubigo-lang Borrow Checker}	
    	\subsection{Borrow Checking Rules}
 		TODO

	\subsection{Example explination}
 		TODO
    	
	\subsection{Solution compared to requirements}
 		TODO



    \section{Rubigo-lang LLVM backend}	
    	Currently the LLVM backend has not been implemented for Rubigo-lang.
	


    \section{Overall course goals and learning outcomes}
    	I have learnt a lot from implementing a parser and EBNF because I encountered a lot of problems when doing so. I learnt a lot about lexical analysis from those problems, like the order the tokens are parsed is very important. And a good way to order them is by longest parser first. I didn't learn as much about syntax analysis because my compiler doesn't support context free grammar.

	From implementing the type checker I feel i haven't learnt a lot, execpt how complicated it can be to organize the symbol table and keep track of all the identifiers. But I have definitely learnt a lot from making the borrow checker because it was very hard to implement and I have never used Rust before. So I learnt a lot about pointers and how Rust borrow rules work.

	The interpreter also made me learn a lot because of how hard it can be to keep track of all the scopes, variables and functions. It made me think of how i could improve my parser, typechecker and borrowchecker so that it could interpret more complex features like pointers.

	Overall I think I have learnt a lot about what not to do when designing a compiler and what the challenges are. But I fell that I haven't learned a lot about the theory and solutions to those hard and complex problems. I have also learnt a lot about Rust and how borrowing works.

	
\end{document}

